\section{Conclusion}
% Claim, evidence, reasoning
In these two experiments, we investigated how the rates of two chemical reactions changed: Calcium carbonate with Hydrochloric acid, and Citric acid with Sodium bicarbonate. We can conclude that increasing concentration, temperature increased the rate of the reactions, while increasing the surface area led to mixed results.

Using the data collected, we could see that the average rate of the first reaction with 3.0 M \ce{HCl} had a much higher magnitude than with 1.0 M \ce{HCl}. $$ (|-0.26 \defunits| > |-0.1 \defunits|) $$ The data collected in the second reaction also shows a higher magnitude of average rate when the denture tablets were placed in higher temperature water. The average rate when the denture tablet was crushed diverged in the two results. In our data, the crushed tablet fully reacted faster than the whole tablet, but in Abigail \& Jersey's data, the crushed tablet had a lower average rate magnitude than the control test.
\begin{align*}
    |\text{Average Rate}_{\text{Heated}}| &> |\text{Average Rate}_{\text{Crushed}}| > |\text{Average Rate}_{\text{Control}}| \\
    |\text{Average Rate}_{\text{Heated (AJ)}}| &> |\text{Average Rate}_{\text{Control (AJ)}}| > |\text{Average Rate}_{\text{Crushed (AJ)}}|
\end{align*}
This may be due to the errors discussed in the following section. The increase in rate with increased concentration and temperature can be explained with collision theory. For the first reaction, tripling the concentration of \ce{HCl} results in more \ce{HCl} particles in the same volume, increasing the frequency of collisions between \ce{HCl} and \ce{CaCO3} particles, thus increasing the rate of reaction. In the second reaction, heating the water increased the kinetic energy of the particles, leading to more particles reaching the activation energy (hitting hard enough), and a faster overall rate. This is coupled with an increase in frequency due to the particles moving more, but this is very minor compared to the former.

In theory, increasing the surface area of a reactant should increase the reaction rate, as the greater area results in more particles being exposed to the other reactant, therefore increasing the frequency of the collisions. The data collected from these experiments, however, is inconclusive on the effect of surface area.


\subsection{Sources of Error} \label{subsec:sources-of-error}
\subsubsection{Major Possible Sources of Error}
For our reactions, we recorded data using a scale with only one decimal place of precision, which severely limited our ability to analyze the denture tablet reactions. This resulted in one of our calculated average rates being 0.0 $\defunits$, which suggests that the reaction was not happening despite qualitative observations saying otherwise. This also made our calculated average rates for the crushed and heated tests equal, despite one finishing faster than the other.
\\~\\
To supplement our low precision data, we used data collected by Abigail \& Jersey, who performed the same experiment but recorded their data with two decimal places of precision. This data was also flawed in that the mass of the system fluctuated throughout the experiment and was observed to increase at a couple points. The system mass in reduction experiments, should not do this, and should instead monotonically decrease.

There are a few possible explanations for this. One is that air currents in the lab, either from movement of people or drafts from windows resulted in various amounts of pressure on the scale. Another major source is that not all of the \ce{CO2} produced by the reaction escaped the system, due to it being trapped in the large amounts of foam observed in all the tests.
\\~\\
A source of error contributing to actual reaction time could be the mixing of the crushed denture tablet. When poured into the beaker, some portions of the powder rested on top of the foam, preventing reaction as the two compounds needed to be aqueous to react.

\subsubsection{Minor Possible Sources of Error}
Two minor possible sources of error for the weight fluctuations could be miscalibration of the scale, and baseline sensor noise. It is difficult to gauge the effect of these sources of error.
\\~\\
Another minor source of error is inconsistent containers and solution amounts used in the Calcium carbonate reaction. We used a graduated cylinder to measure the reaction rate of the 1.0 M \ce{HCl} reaction and a beaker for the 3.0 M \ce{HCl} reaction. This is because we had less 1.0 M \ce{HCl} solution available, and a thinner container was required to fully cover the \ce{CaCO3} sample. I expect this to be a negligible source of error, as the \ce{CaCO3} sample was fully submerged in both cases.
\\~\\
The \ce{CaCO3} samples not being identical could have also contributed to error, but I expect this to also be negligible when looking at overall reaction rate changes. This could have been mitigated by sanding the rocks samples down to known sizes, or crushing into a powder like the denture tablet test. This was originally planned, but determined it was not necessary to measure overall change and not specific change.

\subsection{Procedure Reflection}
While our procedure was sufficient to investigate the effects of concentration and temperature, conclusive data on surface area was not available. Many of the minor possible sources of error could be mitigated through simple procedural adjustments (consistent equipment, rock crushing, etc.) and the usage of a higher precision scale. For the major sources of error, a draft shield could be placed around the scale to prevent currents, and a proper calibration routine could be done at the start of the experiment to ensure accuracy. Preventing the \ce{CO2} from being trapped in the foam is more difficult, but could possibly be mitigated through a change in container pressure or the usage of a nonreactive chemical that prevents foam formation.
\\~\\
These changes would allow for more accurate data to be collected and analyzed. Performing the experiment multiple times and averaging the results would also allow for more realistic data to be collected, as outliers would skew the results less.
\\~\\
If more time was given to write the report, the current data could be more thoughtfully analyzed using linear regressions to provide a trend over the data using every point, instead of an average rate calculated with only the start and end points (only two data points).

%% DONE