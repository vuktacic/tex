\section{Analysis}

\subsection{Concentration Test Analysis}

\begin{figure}[H]
    \centering
    \begin{tikzpicture}
        \begin{axis}[
            title={Mass Change Over Time of an Open \ce{CaCO3(s)} and \ce{HCl(aq)} Reaction},
            xlabel={Time (minutes)},
            ylabel={Test 1 System Mass (g)},
            ylabel style={color=blue},
            ymajorgrids=true,
            yminorgrids=true,
            minor tick num=1,
            xmajorgrids=true,
            grid style=dashed,
            xmin=0, xmax=5,
            xticklabel style={/pgf/number format/fixed, /pgf/number format/precision=1},
            ymin=27.0, ymax=29,
            width=12cm,
            height=7cm,
            axis y line*=left,
            legend style={at={(0.03,0.97)}, anchor=north west}
        ]
        
        \addplot[
            color=blue,
            mark=square,
            line width=0.5pt
        ] table[
            x expr=\coordindex*0.5,
            y=Test 1 Mass (g),
            col sep=comma
        ] {data/concentration/timeseries.csv};
        \addlegendentry{Test 1 Mass (1.0M \ce{HCl})}

        \addplot[
            color=blue,
            mark=square,
            line width=0.5pt,
            dashed
        ] coordinates{
            (0, 28.5)
            (5, 27.8)
        };
        \addlegendentry{Test 1 Average Rate}
        
        \end{axis}
        
        \begin{axis}[
            xlabel={Time (minutes)},
            ylabel={Test 2 System Mass (g)},
            ylabel style={color=red},
            xmin=0, xmax=5,
            ymin=44.5, ymax=46.5,
            width=12cm,
            height=7cm,
            axis y line*=right,
            axis x line=none,
            legend style={at={(0.97,0.97)}, anchor=north east}
        ]
        
        \addplot[
            color=red,
            mark=square,
            line width=0.5pt,
        ] table[
            x expr=\coordindex*0.5,
            y=Test 2 Mass (g),
            col sep=comma
        ] {data/concentration/timeseries.csv};
        \addlegendentry{Test 2 Mass (3.0M \ce{HCl})}

        \addplot[
            color=red,
            mark=square,
            line width=0.5pt,
            dashed
        ] coordinates{
            (0, 46.1)
            (5, 44.8)
        };
        \addlegendentry{Test 2 Average Rate}
        
        \end{axis}
    \end{tikzpicture}
    \caption{Change in mass of the system with average rates over the first 5 minutes of data.}
\end{figure}
% rate is g CO2 / min
For the reaction \reactionOne, reduction in weight can be attributed to \ce{CO2} escaping the system, as it is the only gaseous product. The average rate is therefore in $\defunits$~and is represented in the graph by the slope of the dashed lines. This can be verified by calculating the average rate for both tests from the raw data:
\begin{align*}
    \text{Average Rate}_{1.0M} &= \frac{\text{Test 1 Mass}_{5:00} - \text{Test 1 Mass}_{0:00}}{\text{5:00 - 0:00}} =\frac{27.8 \, g - 28.5 \, g}{5 \, \text{min}} = -0.1 \, \defunits \\
    \text{Average Rate}_{3.0M} &= \frac{\text{Test 2 Mass}_{5:00} - \text{Test 2 Mass}_{0:00}}{\text{5:00 - 0:00}} = \frac{44.8 \, g - 46.1 \, g}{5 \, \text{min}} = -0.26 \, \defunits
\end{align*}

Based on the calculated average rates, it is evident that the reaction with higher concentration of \ce{HCl} had a faster rate of reaction (its magnitude is greater). $ |-0.26 \defunits| > |-0.1 \defunits| $ 

\subsection{Temperature \& Surface Area Test Analysis}

\begin{figure}[H]
    \centering
    \begin{tikzpicture}
        \begin{axis}[
            title={Mass Change Over Time of Denture Tablets (Low Precision)},
            xlabel={Time (minutes)},
            ylabel={System Mass (g)},
            ymajorgrids=true,
            yminorgrids=true,
            minor tick num=1,
            xmajorgrids=true,
            grid style=dashed,
            xmin=0, xmax=5,
            xticklabel style={/pgf/number format/fixed, /pgf/number format/precision=1},
            ymin=2.1, ymax=2.7,
            width=12cm,
            height=7cm,
            legend style={at={(1.05,1)}, anchor=north west}
        ]
        
        \addplot[
            color=blue,
            mark=square,
            line width=0.5pt
        ] table[
            x expr=\coordindex*0.5,
            y=Test 1 Mass (g),
            col sep=comma
        ] {data/tempsurfacearea/timeseries.csv};
        \addlegendentry{Test 1 Mass (Control)}

        \addplot[
            color=blue,
            mark=square,
            line width=1pt,
            dashed
        ] coordinates{
            (0, 2.5)
            (2, 2.5)
        };
        \addlegendentry{Test 1 Average Rate}

        \addplot[
            color=red,
            mark=square,
            line width=0.5pt,
        ] table[
            x expr=\coordindex*0.5,
            y=Test 2 Mass (g),
            col sep=comma
        ] {data/tempsurfacearea/timeseries.csv};
        \addlegendentry{Test 2 Mass (Crushed)}

        \addplot[
            color=red,
            mark=square,
            line width=2pt,
            dashed
        ] coordinates{
            (0, 2.5)
            (2, 2.4)
        };
        \addlegendentry{Test 2 Average Rate}

        \addplot[
            color=green,
            mark=square,
            line width=0.5pt,
        ] table[
            x expr=\coordindex*0.5,
            y=Test 3 Mass (g),
            col sep=comma
        ] {data/tempsurfacearea/timeseries.csv};
        \addlegendentry{Test 3 Mass (Heated)}

        \addplot[
            color=green,
            mark=square,
            line width=0.5pt,
            dashed
        ] coordinates{
            (0, 2.5)
            (2.0, 2.4)
        };
        \addlegendentry{Test 3 Average Rate}
        
        \end{axis}
    \end{tikzpicture}
    \caption{Change in mass of the system with average rates over the first 2 minutes of data.}
\end{figure}

\begin{figure}[H]
    \centering
    \begin{tikzpicture}
        \begin{axis}[
            title={Mass Change Over Time of Denture Tablets (Higher Precision)},
            xlabel={Time (minutes)},
            ylabel={System Mass (g)},
            ymajorgrids=true,
            yminorgrids=true,
            minor tick num=3,
            xmajorgrids=true,
            grid style=dashed,
            xmin=0, xmax=3,
            xticklabel style={/pgf/number format/fixed, /pgf/number format/precision=1},
            ymin=2.1, ymax=2.7,
            width=12cm,
            height=7cm,
            legend style={at={(1.05,1)}, anchor=north west}
        ]
        
        \addplot[
            color=blue,
            mark=square,
            line width=0.5pt
        ] table[
            x expr=\coordindex*0.5,
            y=Test 1 Mass (g),
            col sep=comma
        ] {data/tempsurfacearea/alttimeseries.csv};
        \addlegendentry{Test 1 Mass (Control)}

        \addplot[
            color=blue,
            mark=square,
            line width=0.5pt,
            dashed
        ] coordinates{
            (0, 2.57)
            (3, 2.46)
        };
        \addlegendentry{Test 1 Average Rate}

        \addplot[
            color=red,
            mark=square,
            line width=0.5pt,
        ] table[
            x expr=\coordindex*0.5,
            y=Test 2 Mass (g),
            col sep=comma
        ] {data/tempsurfacearea/alttimeseries.csv};
        \addlegendentry{Test 2 Mass (Crushed)}

        \addplot[
            color=red,
            mark=square,
            line width=0.5pt,
            dashed
        ] coordinates{
            (0, 2.49)
            (3, 2.42)
        };
        \addlegendentry{Test 2 Average Rate}

        \addplot[
            color=green,
            mark=square,
            line width=0.5pt,
        ] table[
            x expr=\coordindex*0.5,
            y=Test 3 Mass (g),
            col sep=comma
        ] {data/tempsurfacearea/alttimeseries.csv};
        \addlegendentry{Test 3 Mass (Heated)}

        \addplot[
            color=green,
            mark=square,
            line width=0.5pt,
            dashed
        ] coordinates{
            (0, 2.59)
            (3, 2.20)
        };
        \addlegendentry{Test 3 Average Rate}
        
        \end{axis}
    \end{tikzpicture}
    \caption{Change in mass of the system with average rates for temperature and surface area test over the first 3 minutes of data with 2 decimal places of precision. (Abigail \& Jersey's data.)}
\end{figure}

As with the first test, in the reaction \reactionTwo, any reductions in weight can be attributed to \ce{CO2} escaping the system, being the only gaseous product. The average rate therefore is $\defunits$~and is also represented in the graph by the slope of the dashed lines. We can calculate the average rate over the first two minutes of our data collection and the first three minutes of Abigail \& Jersey's data collection to verify this:
\begin{align*}
    \text{Average Rate}_{\text{Control}} &= \frac{\text{Test 1 Mass}_{2:00} - \text{Test 1 Mass}_{0:00}}{\text{2:00 - 0:00}} = \frac{2.5 \, g - 2.5 \, g}{2 \, \text{min}} = 0.0 \, \defunits \\
\end{align*}
\begin{align*}
    \text{Average Rate}_{\text{Crushed}} &= -0.05 \defunits \\
    \text{Average Rate}_{\text{Heated}} &= -0.05 \defunits \\
    \text{Average Rate}_{\text{Control (AJ)}} &= -0.037 \defunits \\
    \text{Average Rate}_{\text{Crushed (AJ)}} &= -0.02 \defunits \\
    \text{Average Rate}_{\text{Heated (AJ)}} &= -0.13 \defunits
\end{align*}

\begin{align*}
    |\text{Average Rate}_{\text{Heated}}| &> |\text{Average Rate}_{\text{Crushed}}| > |\text{Average Rate}_{\text{Control}}| \\
    |\text{Average Rate}_{\text{Heated (AJ)}}| &> |\text{Average Rate}_{\text{Control (AJ)}}| > |\text{Average Rate}_{\text{Crushed (AJ)}}|
\end{align*}

With the data we have, we can conclude that crushing the denture tablets and heating the water both increased the rate of reaction. In Abigail \& Jersey's data, the heated test had a much faster rate of reaction than the other two tests. This is further corroborated by our data showing that the heated test fully reacted a full minute before the crushed test.
Another interesting observation is that our crushed test had a faster rate of reaction than our control test and fully reacted before the 5 minute time constraint, but in Abigail \& Jersey's data, the crushed test had a slightly slower rate of reaction than the control test. This may be due to experimental error (\hyperref[subsec:sources-of-error]{Section \ref*{subsec:sources-of-error}: \nameref*{subsec:sources-of-error}}).

%% DONE