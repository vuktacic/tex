\section{Data/Observations}

\subsection{Concetration Test}
\subsubsection{Concentration Test Data}
\begin{table}[H]
    \centering
    
    \begin{minipage}[t]{0.46\textwidth}
        \centering
        \caption{Initial Parameters of Open \ce{CaCO3(s)} and \ce{HCl(aq)} Reaction.}
        
        \vspace{0.5em}

        {
            \aboverulesep=0pt
            \belowrulesep=0pt
        \pgfplotstabletypeset[
            col sep=comma,
            string type,
            every head row/.style={
                before row={\midrule},
                after row={\midrule}
            },
            every nth row={1}{before row=\midrule},
            every last row/.style={
                after row={\midrule}
            },
            columns/Parameter/.style={
                column name=\textbf{Parameter},
                column type={|l}, % <--- Adds left vertical line 
                string type
            },
            columns/Test 1/.style={
                column name=\textbf{Test 1}, 
                fixed, 
                column type={|c}, % <--- Adds center vertical line
                precision=1
            },
            columns/Test 2/.style={
                column name=\textbf{Test 2}, 
                fixed, 
                column type={|c|}, % <--- Adds middle and right-most vertical line
                precision=1
            }
        ]{data/concentration/metadata.csv}
        }
    \end{minipage}
    \hfill
    \begin{minipage}[t]{0.46\textwidth}
        \centering
        \caption{Change in Mass Over Time of an Open \ce{CaCO3(s)} and \ce{HCl(aq)} Reaction.}
        
        \vspace{0.5em}
        
        {
            \aboverulesep=0pt
            \belowrulesep=0pt
        \pgfplotstabletypeset[
            col sep=comma,
            string type,
            every head row/.style={
                before row={\midrule},
                after row={\midrule}
            },
            every nth row={1}{before row=\midrule},
            every last row/.style={
                after row={\midrule}
            },
            columns/Time (min:sec)/.style={
                column name=\textbf{Time},
                column type={|c}, % <--- Adds left vertical line 
                string type
            },
            columns/Test 1 Mass (g)/.style={
                column name=\textbf{Test 1 Mass (g)}, 
                fixed, 
                column type={|c}, % <--- Adds center vertical line
                precision=1
            },
            columns/Test 2 Mass (g)/.style={
                column name=\textbf{Test 2 Mass (g)}, 
                fixed, 
                column type={|c|}, % <--- Adds middle and right-most vertical line
                precision=1
            }
        ]{data/concentration/timeseries.csv}
        }
    \end{minipage}
\end{table}

The mass data collected here is the sum of the mass of the \ce{HCl(aq)} solution and the \ce{CaCO3(s)} rock.
\subsubsection{Concentration Test Observations}

\subsection{Temperature and Surface Area Test Data}
\subsubsection{Our Group's Data}
    \begin{table}[H]
    \centering
    \caption{Initial Parameters of Open Denture Tablet Reactions.}
        \vspace{0.5em}
        {
            \aboverulesep=0pt
            \belowrulesep=0pt
        \pgfplotstabletypeset[
            col sep=comma,
            string type,
            every head row/.style={
                before row={\midrule},
                after row={\midrule}
            },
            every nth row={1}{before row=\midrule},
            every last row/.style={
                after row={\midrule}
            },
            columns/Parameter/.style={
                column name=\textbf{Parameter},
                column type={|l}, % <--- Adds left vertical line 
                string type
            },
            columns/Test 1/.style={
                column name=\textbf{Test 1 (Control)}, 
                fixed, 
                column type={|c}, % <--- Adds center vertical line
                precision=1
            },
            columns/Test 2/.style={
                column name=\textbf{Test 2 (Crushed)}, 
                fixed, 
                column type={|c}, % <--- Adds middle vertical line
                precision=1
            },
            columns/Test 3/.style={
                column name=\textbf{Test 3 (Heated)}, 
                fixed, 
                column type={|c|}, % <--- Adds middle and right-most vertical line
                precision=1
            }
        ]{data/tempsurfacearea/metadata.csv}
        }
\end{table}
    \begin{table}[H]
        \centering
        \caption{Change in Mass Over Time of Open Denture Tablet Reactions.}
        
        \vspace{0.5em}
        {
            \aboverulesep=0pt
            \belowrulesep=0pt
        \pgfplotstabletypeset[
            col sep=comma,
            string type,
            every head row/.style={
                before row={\midrule},
                after row={\midrule}
            },
            every nth row={1}{before row=\midrule},
            every last row/.style={
                after row={\midrule}
            },
            columns/Time (min:sec)/.style={
                column name=\textbf{Time},
                column type={|c}, % <--- Adds left vertical line 
                string type
            },
            columns/Test 1 Mass (g)/.style={
                column name=\textbf{Test 1 Mass (g)}, 
                fixed, 
                column type={|c}, % <--- Adds center vertical line
                precision=1
            },
            columns/Test 2 Mass (g)/.style={
                column name=\textbf{Test 2 Mass (g)}, 
                fixed, 
                column type={|c}, % <--- Adds middle vertical line
                precision=1
            },
            columns/Test 3 Mass (g)/.style={
                column name=\textbf{Test 3 Mass (g)}, 
                fixed, 
                column type={|c|}, % <--- Adds middle and right-most vertical line
                precision=1
            }
        ]{data/tempsurfacearea/timeseries.csv}
        }
        \end{table}

The mass data collected here is based on the mass of the denture tablet. The scale was zeroed with the beaker and water. Data was not collected after the tablet was fully dissolved.

\subsubsection{Abigail \& Jersey's Data}
\begin{table}[H]
    \centering
    \caption{Initial Parameters of Open Denture Tablet Reactions.}
        \vspace{0.5em}
        {
            \aboverulesep=0pt
            \belowrulesep=0pt
        \pgfplotstabletypeset[
            col sep=comma,
            string type,
            every head row/.style={
                before row={\midrule},
                after row={\midrule}
            },
            every nth row={1}{before row=\midrule},
            every last row/.style={
                after row={\midrule}
            },
            columns/Parameter/.style={
                column name=\textbf{Parameter},
                column type={|l}, % <--- Adds left vertical line 
                string type
            },
            columns/Test 1/.style={
                column name=\textbf{Test 1 (Control)}, 
                fixed, 
                column type={|c}, % <--- Adds center vertical line
                precision=1
            },
            columns/Test 2/.style={
                column name=\textbf{Test 2 (Crushed)}, 
                fixed, 
                column type={|c}, % <--- Adds middle vertical line
                precision=1
            },
            columns/Test 3/.style={
                column name=\textbf{Test 3 (Heated)}, 
                fixed, 
                column type={|c|}, % <--- Adds middle and right-most vertical line
                precision=1
            }
        ]{data/tempsurfacearea/altmetadata.csv}
        }
\end{table}
    \begin{table}[H]
        \centering
        \caption{Change in Mass Over Time of Open Denture Tablet Reactions.}
        
        \vspace{0.5em}
        {
            \aboverulesep=0pt
            \belowrulesep=0pt
        \pgfplotstabletypeset[
            col sep=comma,
            string type,
            every head row/.style={
                before row={\midrule},
                after row={\midrule}
            },
            every nth row={1}{before row=\midrule},
            every last row/.style={
                after row={\midrule}
            },
            columns/Time (min:sec)/.style={
                column name=\textbf{Time},
                column type={|c}, % <--- Adds left vertical line 
                string type
            },
            columns/Test 1 Mass (g)/.style={
                column name=\textbf{Test 1 Mass (g)}, 
                fixed, 
                column type={|c}, % <--- Adds center vertical line
                precision=1
            },
            columns/Test 2 Mass (g)/.style={
                column name=\textbf{Test 2 Mass (g)}, 
                fixed, 
                column type={|c}, % <--- Adds middle vertical line
                precision=1
            },
            columns/Test 3 Mass (g)/.style={
                column name=\textbf{Test 3 Mass (g)}, 
                fixed, 
                column type={|c|}, % <--- Adds middle and right-most vertical line
                precision=1
            }
        ]{data/tempsurfacearea/alttimeseries.csv}
        }
        \end{table}

\subsubsection{Temperature \& Surface Area Test Observations}